\chapter{Introduction}

\section{Motivation}

Over the past decade, a number of internet services dedicated to `social networking' have been created. These services allow people to share what they are doing with their friends and to see what their friends have done. They also frequently allow users to see the publicly shared content of other users who they find interesting, perhaps because they know them, perhaps because that person is a celebrity, or perhaps because that person consistently shares content that is interesting in some way such as breaking news, funny jokes, or interesting articles.

The popularity of these social networking services has grown tremendously over the past several years, and that growth has been accompanied by a corresponding growth in both the amount of content available on these networks and the number of services available. Facebook, the largest social network, had over 901 million monthly active users at the end of March 2012 and had over 125 billion friend connections between those users\footnote{http://newsroom.fb.com/content/default.aspx?NewsAreaId=22}. Twitter is another popular social network which has over 140 million active users as of May 2012, with more than 383 million users\footnote{http://www.guardian.co.uk/technology/2012/may/15/twitter-uk-users-10m} having created accounts at some point since the service was founded in 2006. Dozens of other smaller social networks have been created as well, aimed both at general audiences and niche audiences.

With this explosion of content and users it has become much more difficult for users to find novel content of interest to them and to find new users to connect with who might consistently produce such content. Because this may negatively impact the user experience it is in the best interests of both users and the operators of social networking services to provide methods to allow users to easily find interesting content and users.

Existing approaches to content and user recommendation (as described in Section~\ref{sec:ContentRecommendationResearch}) have focused on either a network-based approach, which examines the topology of the network and the connections between users, or a content-based approach, which seeks to take semantic and syntactic information from the content itself and use that to recommend other content. In many cases these approaches are still hamstrung by the same issues that vex individual users, namely the vast amount of data that exists in these social networks which can make analysing them a very difficult task.

Most social networks already provide some form of content recommendation to their users, usually in the form of a list of other users that they may know or be interested in which is created based on the connections that the user already has, i.e. a network-based approach. This is a step in the right direction, but it recommends only users, failing to recommend interesting individual pieces of content. On existing social networks where some content is recommended to users, content is usually only recommended in terms of what is popular amongst all users, without considering the individual user's own interests.

The purpose of this dissertation is to provide a novel and more effective method of recommending both users and content to users of social networking services. This is a more challenging task than it may seem at first. The amount of data from which the recommendations must be drawn is huge---more than 50GB in this case---and the small size of the content available provides far less information for determining relevance than in an application such as web search.

This dissertation will use a combination of a network-based approach and a content-based approach since both have been shown to have value. Taking both methods into account is also very helpful in making the process faster because it is possible to recommend both users and content while running only one algorithm.

In developing this method, it is hoped that such a technique could be put to use either directly by a social networking service or by a third party in order to improve the experience for their users by leading them to more interesting and relevant content. Furthermore, existing techniques for evaluating the effectiveness of such recommendations are very lacking, so it is hoped that new techniques can be used to improve the ability to evaluate both this scheme and future research in the field.

\section{Structure}

This dissertation is structured into six chapters, including this introduction describing the motivation behind the project and the problems it is intended to address.

Chapter 2 gives important background on what social networks are, with particular focus on features that most networks have in common. Two networks in particular, Facebook and Twitter, are discussed in depth to show how these generic features are implemented. This also provides insight on which of these features may be useful in recommending content and users. After giving this background information, existing research on social networks and content recommendation is examined as a means of seeing which network features might be useful for developing recommendations and what techniques have already been used.

Chapter 3 describes the general methodology used by this dissertation. The method is described without reference to any particular social network to emphasize that the technique being used can be applied to nearly any social network having the general features described in Chapter 2. The suitability of this method, known as the Co-HITS algorithm, to social network graphs is described, as are the methods used to transform the graph of the social network into a graph suitable for use by this algorithm.

Chapter 4 describes the specific implementation used by this dissertation. It describes how the data about the social network was selected and obtained and it describes how the challenges of the huge size of this data were dealt with. It then describes the implementation of creating the social graph and implementing the Co-HITS algorithm to run efficiently on this large graph.

Chapter 5 describes the results obtained using this method. Evaluating the results is not straightforward since there is not yet a canonical method of evaluating results in this research area. A number of techniques are proposed and the results of the algorithm for each evaluation technique are presented. Additionally, this chapter describes a number of modifications to the basic algorithm for recommendation and shows how each of these variations affects the results.

Chapter 6 concludes the dissertation. This chapter provides a discussion on how the performance of the specific method developed here could be improved and what future work could be done in this area to improve the overall method by modifying it more fundamentally. The dissertation ends with some concluding remarks on the method developed, its efficacy, and its suitability for real-world implementation.
