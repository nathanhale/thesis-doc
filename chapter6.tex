\chapter{Conclusion}

\section{Concluding Remarks}

The goal of this project was to create an effective content recommendation system for social networks, one capable of recommending both interesting users and interesting content. The results certainly indicate that this goal was achieved.

Between 5\% and 40\% of the top 20 users recommended under the various experiments were users that the distinguished users had decided to follow independently 3 years after the data used was collected. And of the remaining users in the list, the user study revealed that most of them were interesting, with as many as 80\% of the recommended users being of interest. While results close to these could have been achieved with existing methods alone, these good results worked well within the Co-HITS algorithm to provide excellent rankings of tweets.

The more difficult task of recommending individual pieces of content was also very successful. Using the coarse method of cosine similarity to compare the tweets recommended from the time when the data was collected to tweets from today which the user has shown interest in suggests that the tweet recommendations have at least some value over random selections. As a human evaluator it is also clear that the recommendations are good simply by examining what the distinguished users are interested in and the style and content of the tweets recommended to them.

The user study revealed that of the top 20 tweets recommended, between 20\% and 60\% of them were interesting, at least for the two users studied. Both users indicated that the recommendations were excellent, with the obvious caveat that the recommendations are based on their interests of three years ago.

Because of the ad hoc nature of many of the datasets used by various studies in this area it is very difficult to compare the results obtained here to the results of other studies. Still, the results compare favourably with the results that one would expect from using such a recommendation system. Retrieving 40\% relevant content within the top 20 results is quite comparable to the results that are expected when using a search engine, for example. The original paper describing the Co-HITS algorithm (\cite{Deng2009}) had a precision at rank 5 of between 0.35 and 0.39 and precision at 10 of between 0.31 and 0.35 for a web search application, for example.

Additionally, most existing research fails to recommend particular tweets, or if they do it is done simply by selecting tweets from amongst the recommended users. This project not only provided very good tweet recommendations, but it also included many tweets from users other than those in the list of recommended users.

It is also possible to subjectively compare the recommendations generated here to those produced by Twitter and sent to users, an effort which began around the time that this project was begun. The recommended users that Twitter provides have always been generated using a technique similar to that used for the initial user scores here, so while the results here were slightly better than those of Twitter, the advantage was not tremendous.

But Twitter recently began sending users personalized recommendations of particular tweets. Simple examination of these recommendations reveals them to be inferior to the results generated here because they frequently are authored by users who are already followed by the user receiving the recommendation and also seem to be based largely on the number of users who have retweeted the tweet, which means that tweets by users with fewer followers will necessarily not be included.

By all measures available, the algorithm and method described here are very good at recommending content on Twitter. Both the background information on social networks from Chapter 2 and the very general nature of the Co-HITS algorithm described in Chapter 3 suggest that the implementation described in Chapter 4 could be expanded beyond just Twitter and instead be applied to many other social networking services. This is a very important piece of future work that should be done in this area, but it is far from the only one.


\section{Future Work}

\subsection{Experimental Variations}

There are a large number of possible experiments on how best to implement the Co-HITS algorithm for recommending content on Twitter which could not be included in this project due to the time constraints of running them.

The most important of these would be to run the algorithm for a much greater number of distinguished users with a much greater number of repetitions of each of the experiments that have already been done. Many of the experiments described in Chapter 5 were run for only a single user, and that small sample size impacts the reliability of the results. Another important experiment that should be expanded is the experimentation on the weighting of the various edge types. Some work was done on this, as described in Chapter 5, but a much more robust set of experiments would be ideal. The best possible way to do this would be if a (large) set of ground-truth relevance judgements were to emerge for some particular dataset. This would then allow the proper weights to be learned much more accurately using a machine learning algorithm.

Other content edges may also be useful to add. In particular, a URL edge which connects tweets containing the same URL---in a shortened URL form using a service such as bit.ly, most likely---to the authors of those tweets would probably useful. It wouldn't be feasible to expand all of the shortened URLs and to compare them that way, but since a given URL will always map to a particular shortened URL (at least for a period of several years) it is not necessary to expand them. Since URLs are one of the more interesting pieces of content to share by virtue of the greater amount of information they convey, this might be a very valuable edge type.

For calculating the initial user scores, the score of the distinguished user is determined by taking the maximum score from amongst the other users and multiplying that maximum score by 1.5. But the value of 1.5 which was used here is based purely on intuition. It would be good going forward to determine whether this is the correct value or whether it should be lower or higher to produce better results.

\subsection{Improvements}

Another important piece of future work would be to improve the initial tweet scores and more generally to improve the integration of information based on tweet content and style into the algorithm. The results presented in Chapter 5 demonstrated that completely removing the content edges had very little impact on which tweets and users had the highest scores, suggesting that there is plenty of room for improvement on the way in which content is included. This might take the form of Latent Dirichlet Allocation to determine the similarity of the tweets for the initial scores, as was done in several existing research projects, for example.

Content could also be integrated more effectively into the content-based edges. For example, it might be possible to incorporate sentiment analysis into the content edges by only creating links between tweets which discussed the content with a similar sentiment. Alternatively, it might be possible to determine the topics of some of the tweets and to connect tweets based on more concrete measures of their topics to avoid creating the large number of content edges which diluted their effectiveness in the experiments done here.

Yet another option might be to perform clustering on users, tweets, or both, and to use this to make the problem more tractable. Content could be recommended only within a particular cluster to other users within that cluster or links could be made between clusters in order to find groups of users or tweets to recommend. A variation on this would be to filter the content or users which are placed into the final graph in order to ensure that they are of high quality.

\subsection{Other Social Networks}

Finally, and perhaps most importantly, it would be valuable to expand the experimentation done here with the Twitter network to some other social network. Given that the data of most other social networks is generally not available to the public this will be a difficult task, but the algorithm used here is general enough that if the data were available then it would be possible to use on any network.

Some things would need to be adapted, of course. For example with Facebook the retweet edges would change to reshare edges and be less prevalent while hashtags would have to be dropped in favour of some other method of linking content since hashtags are not widely used on Facebook. The promising results shown here suggest that adapting to other networks would provide good results.

Research into content recommendation on social networks is still very much in its early stages, so many of the possibilities have not yet been explored. Recommending individual pieces of interesting content, in particular, has seen very little study so far and the results here do an excellent job of recommending interesting tweets and demonstrating that an approach based primarily on the structure of the social graph can be effective at recommending individual pieces of content. Continuing with this further research would do much to build on the very promising results shown in this dissertation. 

