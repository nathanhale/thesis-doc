\begin{abstract}

Social networking services have exploded in popularity over the past several years and the amount of content available on those services has grown correspondingly. With this much available content it is easy for users to be overwhelmed. Thus, it is in the interests of both the service operators and the users of such services to have access to effective methods for recommending novel content and users to connect with.

This dissertation proposes one such method, adapting and extending a method used in web search to the problem of recommending both users and content within social networks. It also demonstrates a procedure for incorporating links that would otherwise render a social graph non-bipartite into a proper bipartite graph, a technique which has never been used in research on social network content recommendation.

It is then demonstrated that this method is effective at recommending both content and users that a particular distinguished user may be interested in. This efficacy is demonstrated through both user evaluation and calculated metrics. A number of different variations on the methods are explored in order to refine the results further.

\end{abstract}